\section{Overall Description}
% Describe the product's context in the larger business or industry setting.  Do
% not include specific features.  Give the reader an understand of how those
% features fit into the larger setting. 
%

\subsection{Product Perspective}
% How does the product fit in the business larger processes.  How is the user
% intended to fit the software into their business activities?   Consider
% including a figure that illustrates these relationships. 
The Alumni Donation Video Stitcher is projected to increase donations to the 
computer science department of Western Washington University. Currently, 
donations are gathered by sending out emails to alumni with a link to the 
donation site. Our project is going to personalize that process by adding a 
link to a unique video for each alumnus. A week or so before donation emails 
are to be sent out, our client will pass into our video stitching program a 
CSV file with each alumnus's information, name and graduation year, and the 
program will append to that file the links to URLs for each alumnus. The 
stitching will take at most 5 days to process, at which point the office 
who is responsible for sending out donation emails will gain access to the 
CSV file and add the video URL to the email.

\subsection{Product Features}
% This is a list of high-level description of the functional behavior of the
% product.  This should give the reader a better understanding of how the formal
% requirements fit together. 
There are two major components that we are designing: Splitting videos and 
Stitching 
videos.\\\\
Splitting of videos: Will take a directory of unedited videos and will split 
each one on the spaces between words. The split up snippets will be titled by what 
word is being said within, and if the program understands it to be one of our 
accepted words, it will be passed into our edited snippets directory to be accessed 
by the video stitcher. If it is not deemed to be one of our accepted words, it 
will be passed into a pending directory to be viewed and either accepted and 
renamed, or is deleted due to unacceptable content.\\
Stitching of videos: Will parse through the CSV file of graduate students' names 
and graduation years, and for each student randomly select one of our prepared 
sentences and find random snippets from our directory of snippets with precedence to 
snippets taken in the same year as the student graduated. It will then stitch all 
the snippets together to create a approximately 30-second long video, create a URL 
to house it, and hash it. It will then stick that URL in the output file along 
with the alumni's name and graduation year. The output file will then be stored 
in a directory where the office has access. 

\subsection{User Classes and Characteristics}
% Describe the different rolls or classes of users.  For each user class,
% describe the user class's principal characteristics.  For example, unix
% systems have at least two classes of users: system administrators and
% operators.  
\begin{itemize}
    \item Maintainers: are those in charge of maintaining and upgrading 
    the code as need be.
    \item Video collectors: are the people who collect recordings of 
    students and faculty, who then pass it into the video splitter to 
    turn sentences into snippets.
    \item Video stitchers: the person who provides the input CSV file 
    and prompts the program to stitch the videos together.
    \item Video senders: the office who will take the output CSV file 
    and access the video URLs to send to students.
\end{itemize}

\subsection{Operating Environment}
% What is the expected environment?  For example, the product could be a desktop
% application with users who work in a formal office environment.  Contrast this
% with a mobile application for mountain biking that keeps track of GPS locations.
This program is written in python 3.9 with the latest versions of moviepy, 1.0.3, 
and pydub, 0.25.1. It is designed to be run on the university CS lab network to 
deposit and collect videos from in one of our client's Jagodzinski's, directories 
on the network. It is then to be retrieved by the office to email out links.

\subsection{Design and Implementation Constraints}
% List an constraints that are part of the project.  For example, health
% services applications must implement HIPAA regulations.  
\begin{enumerate}
    \item We may only use snippets of people who have submitted a release 
    form and had it accepted.
    \item We or anyone assigned by the client is to record all videos and 
    snippets to ensure quality and security.
    \item To avoid mispronunciations, each student is to record their 
    own name. Meaning, if two people share the same spelling for their 
    names, the snippets will not be used interchangeably. If we do not have 
    a recording of them themselves saying their name, we will instead 
    insert a screen with their name plastered across it for a few seconds.
    \item Since we cannot tie a video snippet of someone's name to a unique 
    artifact like their W\# or their email, we have to rely on their name 
    and graduation year to name the snippet, which may result in students 
    with the same name and graduation year to receive the wrong video.
    \item The user, the person requesting videos to be stitched, will 
    need a machine that runs python 3+ as well as has the latest version 
    of moviepy and pydub
\end{enumerate}

\subsection{Assumptions and Dependencies}
% List assumptions and dependencies that are not formal constraints.  Items in
% this list will, if changed, will cause a change in the formal requirements in
% the next section.
%
\begin{enumerate}
    \item The user, the person requesting videos to be stitched, is doing 
    so on a machine that has access to the video snippets and the directory 
    to where to deposit the output file.
    \item The user is able to run the program.
    \item The office has access to the directory containing the output file.
    \item The alumni are able to open links and view a mp4 of 720 resolution.
\end{enumerate}
