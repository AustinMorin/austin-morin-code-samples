\section{Other Nonfunctional Requirements}
% Nonfunctional requirements are shall-statements about how the software
% performs or is written.  It is not a statement about what the software does.
% For example, the function requirements for a Fibonacci function is this: the
% function shall return the nth Fibonacci number when provided n as input.  A
% nonfunctional requirement would be these: the function shall take time O(n)
% and all lines of the software shall be reachable by some test-case.
%

\subsection{Performance Requirements}
% Your project will have performance requirements.  How long is the user willing
% to wait for the different features to execute?   For example, the software
% shall authenticate authorized users withing 250ms.

The process of stitching the videos together using Python's moviepy and pydub libraries
shall stitch ~10,000 30-second videos in 3 days, where ~3,300 are stitched per day. If we
choose later to implement threading in our program we shall stitch ~10,000 30-second videos in 1.5-2 days, 
where ~4,500 are stitched per day. A single recording session shall take ~30 seconds per student, 
and the transferring of each video file from the Zoomvid-6 camera via USB 3.0 connection to either the Computer 
Science Lab computer or a developer's personal machine shall take a maximum of 1 hour. Sending out all donation 
emails and video URL links to alumni shall be near instantaneous, give or take 5 minutes. Users shall be able 
to access the videos and Computer Science Department Donation Page near instantaneously
given that they have a suitable WiFi connection.

\subsection{Safety Requirements}
% It is unlikely that your project will have any safety requirements.  This
% section would be used for software that controls a physical device that could
% potentially cause harm.  

The use of recording equipment with flashing lights will be avoided to protect epileptic users.

\subsection{Security Requirements}
% Do not overlook this section.  Consider the three primary topics of computer
% security: Confidentiality, Integrity, and Availability. Your project will
% have security requirements. 
%

1) Confidentiality: The development team, Computer Science club members, teacher assistants, and alumni does NOT have access to the
contents of the input CSV files that contain alumni's first names, last names, years of graduation, nor the output CSV files
that contains the first name, last name, graduation year, and the hashed URL to their personalized video.
Dr. Filip Jagodzinski will have maintainer access to the CSV files, and the development team will be responsible for 
creating a video stitching program that he may use to run the CSV files through. The development team has authorized access 
to all video content and URL links.

2) Integrity: The most important factor contributing towards the integrity of the project is the hashing of the URL video links.
This is to prevent possible URL modification with harmful intent such as attackers re-directing alumni to non-trustworthy websites.
The implementation of threading in the stitching program can result in race conditions where audio and video data is lost or overwritten.
To avoid these situations, breaking critical sections up where threads are performed is recommended, and ordering threads to execute once
their predecessor leaves its critical section. The development team is responsible for controlling all of the video content, 
including the pop-up URLs and contact information in the videos, to maintain content consistency and integrity. 
Obsolete and unused CSV and video files will be deleted from Jagodzinski's storage system taking up unnecessary storage space once per 
two academic years. Instructions in two README files shall be written for the stitching program, one will be used to minimize CSV file 
processing errors, and the other will be used to document how to collect the video clips. Regular expressions are used to detect errors 
in CSV file input and output.

3) Availability: The program code shall be engineered to restart upon a system crash, power outage, etc., delete a single that was in 
the process of stitching, and start building the processed video again from scratch. CSV/video files in use during the academic year shall be 
backed-up on a cloud storage system in case the Computer Science Lab Computer file directories are restructured.

\subsection{Software Quality Attributes}
% Quality requirements can be the most difficult to write.  The desired quality
% must be written in a meaningful and testable way.  For example, All functions
% within the software shall have a cyclomatic number less than 10.

The final stitched videos will be ~30 seconds long. The pydub library should take at most 0.5 seconds to extract
all audio spaces from a 30 video recording. Stitching together one video after adding transitions and editing the audio shall take
~25 seconds without threading, and ~10-15 seconds with threading. Reading an input CSV file containing 30 rows shall take less than
1 second, and reading an input CSV file containing thousands of rows shall take at most 30 seconds. For the output CSV files, it shall
take at most 2 seconds to generate and hash 30 video URLs for an input CSV file with 30 rows, and take at most one second to write the
hashed video URLs to a CSV file with 30 rows. Sending a single test email with one video URL link and one CS Donation Page link shall take
less than one second on a satisfactory WiFi connection, and recieving a single test email shall take approximately 5-10 seconds on a satisfactory
WiFi connection.


