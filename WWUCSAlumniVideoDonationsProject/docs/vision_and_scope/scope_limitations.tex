\section{Scope and Limitations}
% This section identifies what part of the envisioned solution is going to be
% developed in this current project or version, what will be delayed until
% latter projects, and what is out of scope for the whole product.  
%
\subsection{Scope of Initial Release}
We are going to develop two distinct programs, 
a video editing program, and a video stitching program.\\ 
The video editing program will accept videos as a (format) 
and parse it for human speech. The program will extract 
snippets where the person is speaking and name each clip 
the word they are saying. For longer clips made up of 
multiple sentences (used for promoting clubs or programs), 
they should already be edited by the submitter and have no 
need to go through the video editing program.\\
For the video stitching program, it will have an assortment of
sentences curated by our team and client that it will randomly
select and build with the clips we've collected. An example
sentence being:\\ 
\{greeting, alumni name, the, computer, science, department, 
would, like, to, ask, you, for, a, donation, to, program message,
thank, you, for, and, take, care\} \\
The program will accept a CSV file where each line contains an
alumni's first name, last name, and year of graduation. 
The program will return a copy of the CSV file with the addition
on each line of a URL to the alumni's personal video. The program
will iterate through the alumni and pick a random sentence
structure for each one, it will then look for clips to make up
the sentence with priority to those taken from the same year 
as the alumnus graduated. It will then stitch all the clips
together, encrypt a URL to store it in, and append it to 
the alumni's line in the CSV file.
If things go well, then we will begin to implement a thank-you 
video for those who donated. It will not be as personal, but 
it will follow the same format of stitching clips together to 
form one of our sentences. Finally, we will add a pop-up of a QR
code in the stitched video that serves as an additional means
for the viewer to access the donation page.
\subsection{Scope of Subsequent Releases}
We envision this project to be released and not updated until our
customer wants to add something to it. At that point they can 
update it themselves, or they can create a new senior project 
team to handle it.
\subsection{Limitations and Exclusions}
The timeframe of our project limits us to only allowing videos 
we ourselves take. In an ideal world, students and alumni would 
be able to submit clips that they would have the creative liberty 
to direct, act, and shoot, but with the variation that would be
submitted our video editing program wouldn't be able to handle it,
resulting in many videos getting rejected, and likely inappropriate
videos getting accepted. The easiest way to address this is to
tightly control what videos can be submitted by recording all of
them ourselves.