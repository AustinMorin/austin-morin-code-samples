\section{Vision of the Solution}
% Now that we know what the customer needs,
% what will the solution look like? 
    
\subsection{Vision Statement}
% This is the formal vision statement.  
%
% For:         user class 
The Western Washington University Computer Science Department
% Who:         statement of need
is looking for a new and engaging form of requesting donations for their programs and clubs from alumni.
% The:         title of product
We have come up with the Alumni Money Video Stitcher,
% Is:          statement of solution
which creates unique videos that stitch various clips of students and staff. The clips will combine to 
greet the alumni by name and request a donation for various aspects of the department, such as 
current projects and clubs that need funding.
% Unlike:      closest alternative solution
The current method for requesting donation is sending generic emails to alumni and asking for money. 
% Our Product: differentiation statement 
Our personalized messages would be more engaging. Seeing their peers, professors, new students, and school projects will hit them with nostalgia, 
which will increase their chances of leaving a donation.
  
\subsection{Major Features}
% Describe the major features that solve the customer's needs.
Our customer is looking for a video made up of separate single word clips to short sentences taken by students and staff. 
They are to be randomly strung together to form an approximately 30-second-long video requesting a donation to the program. 
The video will comprise a greeting, the alumni's name, and a few clips highlighting aspects of the program that need funding. 
It will end off with a request to donate. The video will be stored in an encrypted URL which is then sent to alumni via email.\\
To use the program, the user will simply need to input a CSV file with the first name, last name, and graduation year of each alumnus. 
The program will then stitch together unique videos tailored to each alumnus and spit out a new CSV file containing their 
first name, last name, graduation year, and the encrypted URL to their personalized video.\\
The clips included in each alumnus' video will be chosen at random, but with priority to clips from their graduation year. 
The clip of their name will only be used if we have an instance of the alumnus saying their own name to avoid mispronunciations. 
If we do not have a clip of their name, we will fill that space with a screen with their name typed out.\\
To gather the video clips, we will set up recording sessions and invite students and staff to come in and record clips of 
them saying a list of words and phrases. They will sign release forms and when the forms are processed we will add those clips to our database.

\subsection{Assumptions and Dependencies}
% What does this product depend on?  
% The goal here is to describe, with as much detail as possible, what the envisioned solution needs to operate.  
That students and staff will show up to film videos. If they don't, or we do not receive enough, it will be a 
challenge to stitch together unique videos for each alumnus.
We also require that  alumni will click on links sent to their email, because if we find that alumni will not click our URLs 
then all of this is in vain. There isn't much we can do other than hope they trust our department to send out trustworthy links.