\section{Business Requirements}
\subsection{Background}
% What is the business context for the project?
% What does the reader need to know about the customer's industry?

The overarching goal of this project is to allow Computer Science alumni to have access to a URL link 
in a post-graduation donation e-mail that lets them see a stitched montage video of various students, faculty, and
club representatives asking for donations. Users are provided a link to the WWU Computer 
Science Department's official donation page as well.

The reader needs to know the process of accessing their @wwu.edu email account,
coming to a session on a specified date, time, and location, signing a terms and conditions waiver, 
preparing an elevator pitch for the recording, and following instructions given by the personnel recording
the videos.

\subsection{Business Opportunity}
% What is the customer's need?
% How does this need fit in the industrial context?

The customer's need is to provide a simple means for students to recieve a donation request email upon graduating that
contains a short message, a URL link to a stitched video created from a series of smaller videos of students, faculty, 
and club representatives recorded on campus, a link to a donation page, and a QR within the video as an alternative means to
access a donation page. The ability for students to be able to see the friends that 
they made during their time in the CS department donate and to watch the professors and other faculty present short snippets 
should excite them and encourage them to spread the news on donation opportunities to others.

This need fits into the industrial context since this system will allow students to make donations on a secure site as opposed to
making them via email or on the phone which is more subject to scams and fraud. Furthermore, this system will prevent the student 
from doing more work on their end than they need to and will make the donation experience more engaging and convenient.


\subsection{Business Objectives and Success Criteria}
% Why does the customer need this product?
% How will the customer know that the product is a success?
%   You must be very specific here.  What experiment can we do to verify success?

The customer needs the means to fund the technologies and resources that improve future Computer Science 
students' experiences with assignments, clubs, mentoring sessions, social gatherings, career fairs, 
scholarships, and all other activities associated with the Computer Science Department.

Success can be measured by tracking how much money is donated per year, and compare that amount to how many people
access the video URLs and the link to the offical CS donation page. If the amount of money donated increases from the
previous year, success has been verified.

\subsection{Customer and Market Needs}
% Connect the objectives and success criteria back to the customer's business.
% That is, why does the industry require the customer to have this solution? 

The industry requires the customer to have this solution in order to expand the Computer Science 
Department, provide ease of access to updated technologies and curriculum resources, fund scholarships, 
fund department jobs, and remain relevant to ever-evolving major.

\subsection{Business Risks}
% This is a hazard assessment.  What could go wrong?  How bad would it be if it
% did?  How likely is it?  What steps can we take to protect against the hazard?

Upon initial assessment the project team has identified five hazards:
1) URLs to the video links can be changed. The effects would be negligable since URLs can easily be recovered. 
The chance of this happening is likely.
URL encryption would be required to protect against the hazard.

2) Alumni and faculty do not show up for video recording sessions. 
The effects would be negligable as meetings can be rescheduled on another date. 
The chance of this happening is likely. Steps taken to protect against the hazard can include advertising campaigns, 
incentives such as raffles and gift cards, partnering with CS Clubs, and recruiting additional help with scheduling 
and operating recording sessions.

3) Incorrect info can be inserted into the database of names, e-mails, and URLs by the customer/admin. 
The effects would be serious since it can delay e-mail communication with alumni for a long period of time depending on the damage. 
The chance of this happening is seldom. Steps that can be taken to avoid this problem would be to ensure that all edge cases are covered 
in the program code and make use of regular expressions to identify errors in database insertion.

4) Video files can be corrupted due to a power outage, computer disconnect or shut-down, etc., destroying the URL.
The effects would be serious because this can delay video stitching for a long period of time especially 
if there are a lot of snippets to process. The chance of this happening is occasional. Steps that can be taken to 
avoid this problem would be to use looping logic in the program code to restart the program upon a system crash, delete the video that 
was in-progress, and start building the processed video again from scratch.

5) CS Lab Computer file directory is entirely restructured. 
The effects would be severe because entire files and directories can be lost. The chance
of this happening is seldom. Steps that can be taken to avoid this problem would be to 
use 3rd-party cloud storage such as iCloud, Google Cloud. Backups should not be stored on CS Lab Machine 
and should occur once per quarter.

